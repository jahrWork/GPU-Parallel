\section{Theoretical time}
% \addcontentsline{toc}{section}{Theoretical time}

    \paragraph*{}
    In this section, we discuss the concept of the "theoretical time of a CPU", which refers to the 
    estimated time required for a CPU to complete a given task under ideal conditions. This measure 
    assumes an optimal scenario, free from common real-world limitations such as memory latency, 
    system bottlenecks, or the complexities introduced by parallel execution. By focusing on theoretical 
    performance, we gain insights into the maximum potential of a CPU, providing a useful benchmark 
    for evaluating its capabilities across various workloads.
    \par

    \paragraph*{}
    Theoretical time allows us to break down CPU performance into fundamental parameters, helping us 
    understand how different architectural features influence the speed of computation. This model is 
    particularly valuable for comparing CPUs across generations or architectures, as it highlights 
    the efficiency of vectorization, micro-operations, and core usage, among other factors. While 
    real-world performance is often constrained by a variety of external factors, theoretical models 
    like this one offer a clear, baseline perspective on the CPU's potential.
    \par

    The following equation provides a framework for estimating the theoretical time a CPU would need 
    to complete a specific set of operations:

    \begin{equation}
        t_{CPU}=\frac{N_{ops} \times S_{ops}}{V_{vectorization} \times GH_{z_{CPU}} \times M_{micro-ops} \times C_{CPU}}
        \label{eq:timetheoretical}
    \end{equation}


    Where the parameters are: 
    \begin{enumerate}
        \item \(V_{vectorization}\): Vectorization factor: 16 (512-bit)
        \item \(M_{micro-ops}\): Micro-operations factor: 4, 6 (AMD Zen 3) or even 8 (Apple Silicon)
        \item \(GH_{z_{CPU}}\): Clock speed of the CPU
        \item \(C_{CPU}\): Number of cores in the CPU
        \item \(S_{ops}\): Sequence of operations. For example, in the case of matrix multiplication, it would have a value of 4.
        \item \(N_{ops}\): Number of operations
    \end{enumerate}

\newpage