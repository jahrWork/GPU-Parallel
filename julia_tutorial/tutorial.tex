\documentclass[a4paper,12pt]{article}
\usepackage[utf8]{inputenc}
\usepackage{amsmath,amsfonts,amssymb}
\usepackage{graphicx}
\usepackage{listings}
\usepackage{xcolor}
\usepackage{hyperref}

\title{Introduction to Julia: A Comprehensive Guide}
\author{Author: Mario Alonso Cuero}
\date{September 2024}

\begin{document}

\maketitle

\tableofcontents

\newpage

\section{Installing Julia}

\subsection{Download and Installation}
Julia can be downloaded from its official website \href{https://julialang.org/downloads/}{here}. Select the appropriate version for your operating system and follow the installation instructions.

\subsection{Setting Up the Environment}
We recommend using Julia with Visual Studio Code, which can be downloaded from \href{https://code.visualstudio.com/}{Visual Studio Code}. Install the Julia extension for a better development experience.

\section{Julia Language: Basic Concepts}

\subsection{Basic Syntax}
Julia is a dynamic, high-performance programming language. Here's a basic example of how to define a variable and a function:

\begin{lstlisting}[language=Python]
# Variable definition
x = 10

# Function definition
function sum(a, b)
    return a + b
end

# Calling the function
result = sum(3, 4)
println(result)  # Prints 7
\end{lstlisting}

\subsection{Basic Packages and Libraries}

Julia uses a package manager called \texttt{Pkg}. To install a package, use the following command in the REPL:

\begin{lstlisting}[language=Python]
using Pkg
Pkg.add("PackageName")
\end{lstlisting}

Some recommended packages are:

\begin{itemize}
    \item \texttt{Plots.jl} for plotting.
    \item \texttt{Distributions.jl} for working with statistical distributions.
    \item \texttt{DataFrames.jl} for handling tabular data.
    \item \texttt{BenchmarkTools.jl} for performance benchmarking.
\end{itemize}

\section{Mini Tutorial of Julia with Examples}

\subsection{Basic Operations}

\begin{lstlisting}[language=Python]
# Basic mathematical operations
a = 10
b = 5
c = a * b  # Multiplication
d = a / b  # Division
println(c)  # 50
println(d)  # 2.0
\end{lstlisting}

\subsection{Vectors and Matrices}

\begin{lstlisting}[language=Python]
# Defining a vector
v = [1, 2, 3, 4]

# Defining a matrix
m = [1 2; 3 4]

# Tensor product
t = v * v'
println(t)
\end{lstlisting}

\section{Plotting Graphs and Contour Maps}

\subsection{2D Plots}

\begin{lstlisting}[language=Python]
using Plots

# Plotting y = f(x)
x = 0:0.1:10
y = sin.(x)
plot(x, y, label="sin(x)")
\end{lstlisting}

\subsection{Contour Maps}

\begin{lstlisting}[language=Python]
using Plots

# Defining the function
f(x, y) = sin(x) * cos(y)

# Range of values
x = y = 0:0.1:10

# Contour map
contour(x, y, (x, y) -> f(x, y))
\end{lstlisting}

\section{2D Animations}

\begin{lstlisting}[language=Python]
using Plots

# Setting up the backend
gr()

# Initial data
x = 0:0.1:10
y = sin.(x)

# Create the animation
anim = @animate for i in 1:100
    plot(x, sin.(x .+ 0.1*i), ylim=(-1,1))
end

# Save the animation
gif(anim, "sine_wave.gif", fps=10)
\end{lstlisting}

\section{Encapsulating Functions for Plotting}

\begin{lstlisting}[language=Python]
function plot_function(f, x_range)
    y = f.(x_range)
    plot(x_range, y, label="f(x)")
end

# Usage example
plot_function(x -> x^2, -10:0.1:10)
\end{lstlisting}

\section{Functions, Vectors, Matrices, and Tensor Product}

\subsection{Functions}

\begin{lstlisting}[language=Python]
# Basic function
f(x) = 2x + 3
println(f(5))  # Prints 13
\end{lstlisting}

\subsection{Vectors and Matrices}

\begin{lstlisting}[language=Python]
# Defining a vector
v = [1, 2, 3]

# Defining a matrix
m = [1 2 3; 4 5 6; 7 8 9]
\end{lstlisting}

\subsection{Tensor Product}

\begin{lstlisting}[language=Python]
using LinearAlgebra

# Tensor product of vectors
v = [1, 2]
t = kron(v, v)
println(t)
\end{lstlisting}

\section{Benchmarking and Measuring Time in Julia}

\subsection{Basic Timing with @time}
Julia provides the `@time` macro to measure the execution time of code blocks. Here's an example:

\begin{lstlisting}[language=Python]
# Basic timing
@time begin
    sum = 0
    for i in 1:1000000
        sum += i
    end
    println(sum)
end
\end{lstlisting}

\subsection{Accurate Timing with @btime from BenchmarkTools}

For more accurate benchmarking, use the `@btime` macro from the `BenchmarkTools.jl` package:

\begin{lstlisting}[language=Python]
using BenchmarkTools

# Accurate timing
@btime sum(1:1000000)
\end{lstlisting}

\subsection{Benchmarking Functions}
You can also benchmark custom functions:

\begin{lstlisting}[language=Python]
function my_function(n)
    sum = 0
    for i in 1:n
        sum += i
    end
    return sum
end

@btime my_function(1000000)
\end{lstlisting}

\section{Modules and Inclusion Methodology}

\begin{lstlisting}[language=Python]
module MyModule

export my_function

my_function(x) = x^2

end

# Using the module
using .MyModule

println(my_function(4))  # Prints 16
\end{lstlisting}

\section{GitHub Integration with Visual Studio Code}

\subsection{Initial Setup}

First, install Git on your system and clone your GitHub repository in Visual Studio Code:

\begin{lstlisting}
# Cloning a repository
git clone https://github.com/username/repository.git
\end{lstlisting}

\subsection{Basic Usage}

Within Visual Studio Code, you can commit, push, and pull changes either from the graphical interface or the integrated terminal:

\begin{lstlisting}
# Add changes and commit
git add .
git commit -m "Commit message"
git push origin main
\end{lstlisting}

\section{Best Practices in Julia}

\begin{itemize}
    \item Use descriptive names for variables and functions.
    \item Write clear and concise comments.
    \item Prefer vectorization over loops for operations on arrays.
    \item Organize your code into modules to improve maintainability.
    \item Use \texttt{Revise.jl} for more efficient interactive development.
\end{itemize}

\section{GPU-Parallel}

To use the package `GPU-Parallel`, first, you need to install Julia using `juliaup`. For Windows users, this can be done via the Microsoft Store.

After installing Julia, it is highly advisable to install the following packages in the global environment:

\begin{lstlisting}[language=Python]
using Pkg
Pkg.add("Revise")
Pkg.add("BenchmarkTools")
Pkg.add("ProfileView")
\end{lstlisting}

Then, to activate the local environment and install the necessary packages, run the following commands in a Julia REPL:

\begin{lstlisting}[language=Python]
using Pkg
Pkg.activate(".")
Pkg.instantiate()
\end{lstlisting}

With this setup, you'll be ready to develop and run GPU-accelerated parallel code in Julia.

\end{document}
