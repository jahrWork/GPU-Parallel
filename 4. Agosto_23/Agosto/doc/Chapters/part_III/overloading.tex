
\chapter{Overloading functions and operators}    \label{ch:overloading}
From the mathematical point of view operators and functions representing 
common concepts can be to different types of different n-dimensional spaces.
It is desirable to maintain the same operators or the same function names 
by overloading function names and operator symbols. 
If the programming language is a compiling language, such as Fortran, 
the compiler identifies, at compilation time, the type of the operands or the arguments 
involved when calling a function and it selects automatically the proper operation. 
If the language is untyped interpreted language, such as Python, overloading is not supported. 
However, there are many alternatives to emulate overloading in Python. 

Operators can be applied to elements of sets with different types.
For example, the operator \texttt{+} 
is defined differently for real numbers of complex numbers. However, the same operator 
is used for those different operations. This is done by overloading the operator. 
If $x$ and $y$ are reals, the expression $ x+y $ involves the operator \texttt{+}
which is the classical sum. 
However, if  $u$ and $v$ are complexes, 
the expression $ u+v $ involves the same operator \texttt{+}. It this case, 
the result yields a complex number in which 
the real part is the sum of real parts of $ u $ and $ v $
and the imaginary part is the sum of imaginary parts of $ u $ and $ v $.
Many operands and functions are previously overloaded by the native language.

Additionally to operators, functions can also be overloaded. 
Functions are overloaded with n-dimensional spaces. 
For example, the mathematical concept: integral 
can be defined and overloaded for line integrals or surface integrals. 
 

\newpage 
\section{Overloading functions in n-dimensional spaces}

In this example, the function \texttt{Integral} is created to overload line integrals 
and surface integrals. 
Let's consider the  line integral in a compact segment $[a, b]$: 
\begin{equation}   
    I =  \int _a ^b f(x) \ dx, \qquad f: \mathbb{R} \rightarrow \mathbb{R},   
\end{equation} 
and the surface integral in a compact square $ \Omega = [a, b] \times [c,d]$:   
\begin{equation}   
    I =  \int _{\partial \Omega} f(x, y) \ dx \ dy, 
    \qquad f: \mathbb{R}\times \mathbb{R} \rightarrow \mathbb{R}.   
\end{equation} 
From the software point of view, using the same word \texttt{Integral}
to invoke surface or line integrals is desirable. 
Besides, a surface integral can be expressed by the following way: 
\begin{equation}   
    I =  \int _{a} ^b \left(   \int _{c} ^d f(x,y) \ dy      \right) \ dx,
\end{equation} 
in which the integrand is defined with the parametric line integral:
\begin{equation}   
   I_1(x) = \int _{c} ^d f(x,y) \ dy.  
   \label{I1}   
\end{equation} 
The final expression for the surface integral is:  
\begin{equation}   
    I =   \int _{a} ^b I_1(x) \ dx. 
    \label{I2D}  
\end{equation} 
Hence, a surface integral can be defined by functional composition of line integrals. 
In the expression,  $ I_1(x) $, the variable $ x $ acts as a parameter
and $ I_1(x) $ is calculated by means a line integral. 
 
In the following pages, these two concepts overloading and functional composition
are taken into account to write two implementations in Fortran and Python. 

 
\newpage  
\subsection*{Fortran}

\renewcommand{\home}{./Fortran/sources/Advanced_programming/overloading} 
\lstfor

First, the overloading implementation is explained. Once two different functions are 
created for line and surface integrals, the overloading implementation is done
with the following code: 
\vspace{0.5cm}  
\listings{\home/overloading.f90}{interface Integral}{end interface}{overloading.f90}
At compilation time, function invocations are analyzed ans its proper function
is identified. In the following code, two integrals are invoked with 
different number and type of arguments.
\vspace{0.5cm} 
\listings{\home/overloading.f90}{subroutine test_Integral}{end subroutine}{overloading.f90}
The first \texttt{Integral} invocation has three arguments, the two first are reals and the 
third \texttt{f1} is a real function.
The second \texttt{Integral} invocation has five arguments, the four first are reals and 
the fith \texttt{f2} is a real function 
$(\mathbb{R}\times \mathbb{R} \rightarrow \mathbb{R})$.
What rest is to write, \texttt{Integral1D} and \texttt{Integral2D} accoding to their 
specifications.  
\newpage  

Let's begin with  \verb|Integral1D| implementation. 
Even though the numerical method to approximate the line integral is not relevant, 
a Riemann approximation is shown in the following code to have a complete testing
example that could be used as a template. 
\vspace{0.5cm} 
\listings{\home/overloading.f90}{function Integral1D}{end function}{overloading.f90} 
Note that the third argument is defined as a \verb|f_R_R| procedure or function. 
Hence, the definitions of \verb|f_R_R| for line integrals and \verb|f_R2_R|
for surface integrals are given in the following code: 
\vspace{0.5cm} 
\listings{\home/overloading.f90}{abstract interface}{end interface}{overloading.f90} 
 
\newpage  
Let's finish with \verb|Integral2D| implementation. 
Taking into account the definition of a surface integral expressed 
by  equations (\ref{I2D}) and (\ref{I1}), the following implementation is clear: 

\vspace{0.5cm}  
\listings{\home/overloading.f90}{function Integral2D}{end function}{overloading.f90} 
\verb|Integral2D|  is built by a line integral (\verb|Integral1D|) 
in which the integrand is parametric line integral defined by: 
\vspace{0.5cm}   
\listings{\home/overloading.f90}{function Parametric_I1D}{end function}{overloading.f90} 
    
 
 
\newpage
\subsection*{Python}
\renewcommand{\home}{./Python/sources/Advanced_programming/overloading}
\lstpython 
The same concepts are applied to implement an overloaded \verb|Integral| in Python. 
As it is mentioned repetitively, Python allows writing shorter codes 
than Fortran but paying the price of no specifying function arguments or variables. 


Even though Python does not allow overloading, an easy proposa  is shown to 
emulate overloading in the following code: 
\vspace{0.5cm} 
\listingsp{\home/overloading.py}{def Integral }{None}{overloading.py} 
At execution time, the \verb|Integral| function 
checks how may arguments are present. If the invocation is done with three
arguments, line integral is assumed. 
 If the invocation is done with five
arguments, surface integral is assumed. Otherwise, \verb|Integral| function
will show an error.  

In the following code, an example of the overloaded  \verb|Integral| is shown: 
\vspace{0.5cm}
\listingsp{\home/overloading.py}{def test_Integral}{y}{overloading.py}
Note that the first invocation has three arguments and the second invocation has five 
arguments. Note also that anonymous functions or lambda functions have been 
expressed in the last argument to make the example more compact.  


\newpage
Finallly, \verb|Integral1D| and \verb|Integral2D| ae implemented according to their
definitions.
\verb|Integral1D| is approximated by Riemann sums 
\vspace{0.5cm}
\listingsp{\home/overloading.py}{def Integral1D}{return}{overloading.py}

And \verb|Integral2D| is expressed by means of \verb|Integral1D| function 
according to its definition given by (\ref{I2D}) and (\ref{I1}). 
\vspace{0.5cm}
\listingsp{\home/overloading.py}{def Integral2D}{a, b, I1}{overloading.py}



\newpage 
\section{Overloading functions with complex numbers}

In this example, an ordinary differential system of equations is integrated
in real variable and complex variable. 
Let's consider the first order 
differential system of equations of dimension $ N $ in real variable: 
\begin{equation}   
    \frac{dU }{ dt} =  F( U,t ) , \qquad 
    F: \mathbb{R}^N \times \mathbb{R} \rightarrow \mathbb{R}^N,   
    \label{cauchy}
\end{equation} 
where $ U(t) \in \mathbb{R}^N $. This system of equations together with 
the initial condition $ U_0 $ constitutes a Cauchy problem and allows 
to determine the solution $ U(t) $ for all $ t \in [0, T]$.
To approximate numerically the solution of the Cauchy problem, a partition of
the time segment $ [0, T] $ is considered. 
Once this partition is defined by different time steps $ t_n $, the solution 
can be approximated by any temporal scheme. 
The Euler scheme is the easiest numerical method to approximate numerically 
the solution $ U^{n+1} $ at time step $t_{n+1} $ from the solution 
$ U^{n} $ at time step $t_{n} $. 
\begin{equation}   
    U^{n+1}=  U^n + \Delta t \ F( U^n, t_n), \qquad 
    n=0, \ldots, N  
    \label{euler}   
\end{equation}
In this example, 
the harmonic oscillator ($ \ddot x   + x = 0 $)  is integrated.
This oscillator can be formulated as a first order system of two equations:
\begin{equation}   
    \frac{d}{dt }   
 \begin{pmatrix}
 x \\
 \dot x
 \end{pmatrix}   
    = 
  \begin{pmatrix}
  \dot x \\
  - x
  \end{pmatrix}
  \label{oscillator}
\end{equation}
where the state vector can be  defined with $ U = ( x, \ \dot x  )^T $ 
and, consequently,  $ F= ( U_2, \ -U_1  )^T $. The same harmonic oscillator
can be expressed in complex variable by doing  $ Z = x + i \ \dot x  $. 
In complex variable, the equations of the harmonic oscillator yields: 
\begin{equation}   
    \frac{dZ}{dt }  = - i \ Z.
    \label{coscillator}
\end{equation} 
The main objective of this section is to show different alternatives 
to integrate in complex variable by means of functions written for 
real variable. 

Since Python is untyped, a minor modification will allow to integrate 
in real or in complex variable with the same functions.
On the contrary, Fortran variables are obliged to be typed.  
In this case, instead of writing new software to overload the Euler 
scheme, a different approach will be shown. The real function $ F $ in equation
(\ref{cauchy}) will be build by means of the complex function of equation
(\ref{coscillator}) and the complex system of equations will be integrated in real 
variable with minor modifications. 

In the following pages, these ideas will be developed 
to write two implementations in Fortran and Python. 

 
\newpage  
\subsection*{Fortran}
First, the \verb|Euler| function 
is implemented for real variable. Given an approximate solution \verb|U1| 
at time step \verb|t1|, the solution \verb|U2| at  time step  \verb|t2| 
is calculated by (\ref{euler}). 
\vspace{0.4cm}  
\renewcommand{\home}{./Fortran/sources/Advanced_programming/polymorphism} 
\lstfor
\listings{\home/polymorphic_ODES.f90}{function Euler}{end function}{polymorphic_ODES.f90}
Then, the real function of the harmonic oscillator (\ref{oscillator}) is defined by:
\vspace{0.4cm} 
\listings{\home/polymorphic_ODES.f90}{function oscillator}{end function}{polymorphic_ODES.f90}
Finally, a loop allows to calculate and to store every time 
step of the time partition. The matrix \verb|U(0:N,Nv)| holds \verb|N+1|
time steps for every \verb|Nv| variables. 
\vspace{0.4cm} 
\listings{\home/polymorphic_ODES.f90}{simulation}{end do}{polymorphic_ODES.f90}


\newpage 
To integrate the complex system, 
the harmonic oscillator equations (\ref{coscillator}) written in complex 
variable are implement with the following function: 
\vspace{0.4cm} 
\listings{\home/polymorphic_ODES.f90}{complex_oscillator}{end function}{polymorphic_ODES.f90}
Finally, the function \verb|complex_simulation| is built with complex 
arguments. It defines a real function \verb|F| which is created 
by means of the complex function \verb|Fc| and then 
the problem can be simulated in real variable. 
\vspace{0.4cm} 
\listings{\home/polymorphic_ODES.f90}{complex_simulation}{end subroutine}{polymorphic_ODES.f90}

 
\newpage
\subsection*{Python}
As it was mentioned, to overload the Python code to integrate a complex problem 
is much easier because Python functions are implemented without knowing 
the type of variables. 
First, the \verb|Euler| function 
is implemented. Given an approximate solution \verb|U1| 
at time step \verb|t1|, the solution \verb|U2| at  time step  \verb|t2| 
is calculated by (\ref{euler}).
\vspace{0.4cm} 
\renewcommand{\home}{./Python/sources/Advanced_programming/polymorphism}
\lstpython 
\listingsp{\home/polymorphic_ODES.py}{def Euler}{return}{polymorphic_ODES.py.py}  
Then, the real function of the harmonic oscillator (\ref{oscillator}) is defined by:
\vspace{0.4cm} 
\listingsp{\home/polymorphic_ODES.py}{def oscillator}{return}{polymorphic_ODES.py}
Finally, a loop allows to calculate and to store every time 
step of the time partition. The matrix \verb|U(0:N,Nv)| holds \verb|N+1|
time steps for every \verb|Nv| variables. 
\vspace{0.4cm} 
\listingsp{\home/polymorphic_ODES.py}{def simulation}{Euler}{polymorphic_ODES.py}
To integrate the system in complex variable, 
the complex function of the harmonic oscillator (\ref{coscillator}) is defined by:
\vspace{0.4cm} 
\listingsp{\home/polymorphic_ODES.py}{def complex_oscillator}{return}{polymorphic_ODES.py}
\newpage 
In the following code, the harmonic oscillator is integrated in real and in complex variable with the same 
code: 
\vspace{0.5cm} 
\listingsp{\home/polymorphic_ODES.py}{def complex_ODES}{complex_oscillator}{polymorphic_ODES.py} 
Note that while \verb|U0| is a nump array of real components, \verb|Z0| is a complex variable holding the 
same initial condition than \verb|U0|. 
The key idea of this overloading is hidden in the function \verb|simulation|. When this function is called, 
a numpy array to store different time steps of all variables is created. The type of its components 
is the same as the type of the initial condition. Hence, if the \verb|simulation| function is called 
with a complex initial condition, the numpy array \verb|U| has complex components and different operations are 
well defined. Similarly, if the initial condition has real components, the numpy array \verb|U| 
is created with real components and everything works flawlessly as expected. 