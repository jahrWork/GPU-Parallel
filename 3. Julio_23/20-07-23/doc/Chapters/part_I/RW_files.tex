
\newpage   
\chapter{Reading and writing external files}    \label{chap:readwrite}



Reading and writing data from/to external files is usually an essential part of every software project. Maybe you want to process data returned by a different software, share information between two programs written in different languages, process the results from your program in an specialised plotting software or just reading the inputs of your program through a data sheet. 

The aim of this chapter is to propose an orderly way to encapsulate and use a matrix reading function. As a result, this function hides the particularities of Fortran (or any other programming language where imitating this function) related to reading/writing information from files and builds an useful tool that can be easily recycled for future projects. A one-time effort to create a powerful and general tool is worthwhile if you are using a programming language regularly. Additionally, encapsulating and separately validating this tool reduces failures and helps to standardize the way of proceeding each time that a reading operation is performed.

The following subroutine loads two matrices by declaring the relative path of the \texttt{.csv} file where information is stored and then writes them in the screen by rows. Notice that the dynamic object \texttt{A} in the subroutine does not need to be allocated since the compiler automatically takes care of that when the output of the reading function is available. Also, notice that the matrix is written using an explicit loop for rows and an implicit loop for columns. This technique is preferred since Fortran writes by default the data by columns. 

\newpage 
\vspace{0.5cm} 
\renewcommand{\home}{./Fortran/sources/Foundations/read files} 
\listings{\home/read_files.f90}{subroutine Test_load_matrix}
{end subroutine}{read_files.f90}


%-------------------------------
The function \texttt{load\_matrix()} is in charge of reading the data and assigning it to a local variable \texttt{A}. Contrary to the subroutine, here the dynamic object \texttt{A} is allocated by hand. 

In summary, this function opens the requested file (input argument), reads the first line and saves the information in the variable \texttt{header}. This header determines the number of columns in which the data is ordered (using commas as separators). From now on, all rows are read until the total number is determined and the size of array \texttt{A} can be defined. Finally, the whole file can be reread again, saving the data in \texttt{A} this time. This function makes use of an auxiliary function called \texttt{columns( string, s )} which reads how many columns are declared in the header of the file using some specific separator. 

Encapsulating some tools like this one allows the programmer to save time and reduce possible sources of error in future projects but does not eliminate the need to know the peculiarities of reading/writing data in files or screen, as is the case with any programming language. For the needs of each project some adjustments could be done.

\newpage
\vspace{0.5cm} 
\renewcommand{\home}{./Fortran/sources/Foundations/read files}
\listings{\home/read_files.f90}{function load_matrix}
{end function}{read_files.f90}

\renewcommand{\home}{./Fortran/sources/Foundations/read files}
\listings{\home/read_files.f90}{function columns}
{end function}{read_files.f90}

  
  
  
  
\newpage 
\subsection*{Python code} 
The following examples reproduce the same behaviour with Python, notice that many functions are already implemented in different packages of Python. 
In this case the function \texttt{load\_matrix()} encapsulates the load of the specific file through \texttt{read\_csv()} (included in Pandas package), prints the header and returns the needed values. As mentioned before, encapsulating tools like this one allows you to save time and reduce possible sources of error in future projects by creating an standard for all your programs. 


\vspace{0.5cm}
\renewcommand{\home}{./Python/sources/Foundations/Read_files}
\lstpython 
   \listingsp{\home/Read_files.py}{def Test_load_matrix}
    {end}{Read_files.py}
  
 \renewcommand{\home}{./Python/sources/Foundations/Read_files} 
    \listingsp{\home/Read_files.py}{def load_matrix}
     {return}{Read_files.py} 
  \lstfor
  
  
  
 %   A batch file named: \mycap{Preliminary_examples.bat} allows to compile and to run this main program.  
 
 
 % \newpage  
 
 
 
 
 %   \subsection{Windows}
 
 
 % _______________ RUN EXAMPLES_________________________________________________
 
 %   >compile_Cauchy_Problem_examples.bat
 %   
 %   >compile_BVP_examples.bat 
 %   
 %   >
 %   
 %   >
 %   
 %   
 %   
 %   Examples of use compile.bat 
 %   
 % %  _________BASIC USAGE___________________________________________________________ 
 %   
 % %  >compile main 
 %   
 %   The file main.f90 contains the main program to be executed
 %   
 %%   e.g. to compile and to execute the "hello world" first example 
 %   
 %   >compile main    
 %   
 %   
 % %  ___________ADVANCED USER______________________________________________________
 %   
 %   If the user develops a new module, or he wants to use existing examples,
 %   the compilation procedure is as follows: 
 %   
 %%   > compile module_name main 
 %   
 %   1) module_name is a fortran module file which is located in sources. 
 %   
 %   2) main file is the main program.
 %   It contains a use setence to access module_name and it calls some subrutine of  the module 
 %   
 %   
 %   e.g. 
 %   > compile 
 %   
 
 
 
 
 
 
 %   
 %   
 %   
 %   
 %   Informatics 2017/2018 Aerospace Engineering (UPM) 
 %   
 %   How to install FORTRAN and graphical windows in windows 10 
 %   
 %   Author: Juan A. Hernandez:  juanantonio.hernandez@upm.es 
 %   
 %   
 %   
 %   %   A) Copy Informatica\_2018 to Desktop 
 %   
 %   B) Check the compiler 
 %   1)open command prompt and type >compile Hello\_world
 %   2)open  command prompt and type >compile plot\_graph
 %   
 %   
 %   \subsection{Ubuntu}
 %   
 %   
 %   
 %   Informatics 2017/2018 Aerospace Engineering (UPM) 
 %   
 %   How to install FORTRAN and graphical windows in Ubuntu from windows 10 
 %   
 %   Author: Juan A. Hernandez:  juanantonio.hernandez@upm.es 
 %   
 %   
 %   
 %   
 %   A) Install UBUNTU from Windows 10 (skip this step if your machine is UBUNTU) 
 %   
 %   1) Setting/Security/Developer ( allow to create and modify programs ) 
 %   
 %   2) control panel/programs/Windows characteristics allow UBUNTU 
 %   
 %   3) cmd> bash . Create user and password UBUNTU machine 
 %   
 %   4) Install X11 server for graphical windows ( XMing ) to run graphical linux binaries from shell.
 %   Capability to start 64-bit GUI-programs from the desktop file manager 
 %   
 %   5) Install graphical vim editor to check: 
 %   
 %   \$sudo apt-get install vim-gtk
 %   
 %   Now, you?ll need to set the ?DISPLAY? environment variable to point at the X server running on your Windows 10 PC. 
 %   If you don?t do this, graphical applications will simply fail to launch.
 %   To do this, run the following command in the Bash environment:
 %   
 %   \$export DISPLAY=:0
 %   \$gvim
 %   
 %   
 %   B) Install gfortran 
 %   
 %   1) \$ sudo apt-get install gfortran 
 %   
 %   
 %   
 %   C) Install DISLIN to plot with UBUNTU 
 %   
 %   1)  should install OpenMotif on Ubuntu 
 %   
 %   
 %   \$ sudo add-apt-repository deb http://archive.ubuntu.com/ubuntu lsb\_release -sc universe multiverse
 %   
 %   \$ sudo apt-get update 
 %   
 %   \$ sudo apt-get install libxm4
 %   
 %   \$ sudo apt-get install libmotif4* libmotif-dev
 %   
 %   \$sudo apt-get install libx11-dev libxt-dev libgl1-mesa-dev  WARNING:(l1 ele and then one )
 %   
 %   
 %   2) Check the compiler 
 %   
 %   1)\$ ./compile.sh Hello\_world
 %   2)\$ ./compile.sh plot\_graph
 %   
 %   
 %   
 %   \subsection{MAC} 
 %   
 %   Informatics 2017/2018 Aerospace Engineering (UPM) 
 %   
 %   How to install FORTRAN and graphical windows in MAC OS 
 %   
 %   Author: Juan A. Hernandez:  juanantonio.hernandez@upm.es 
 %   
 %   
 %   1) How to execute a Terminal 
 %   
 %   Applications/ Utilities/ Terminal
 %   
 %   2) How to execute a Terminal in a specific folder
 %   
 %   System Preferences/ Keyboard/ Speed functions/ Services - scroll and select "Terminal in the folder"
 %   
 %   
 %   
 %   3) Install gFortran to compile
 %   
 %   3.1) Install manager HOMEBREW (https://brew.sh)
 %   3.2) \$ brew install gcc
 %   
 %   
 %   4) Install Openmotiff
 %   
 %   4.1) Install server X11 through XQuartz (https://www.xquartz.org)
 %   Capability to start 64-bit GUI-programs using x-windows from the shell 
 %   
 %   4.1)\$ brew install openmotif
 %   