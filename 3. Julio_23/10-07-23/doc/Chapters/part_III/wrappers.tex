

\chapter{Wrappers or decorators} 

\vspace{-1cm} 
\section{Introduction} 
\vspace{-0.5cm}
Wrappers or decorators is perhaps one of most powerful  
programming skills to be used by advanced developers. 
It resembles the natural way of thinking.
Let's imagine that our main objective or task is to simulate 
the behavior of a dynamical system. To achieve that goal, libraries 
concerning to numerical schemes to integrate the problem should be written.
The final result is a function to simulate the problem. 
Generally, this software has been implemented in some library and there is no 
need to implement it again. 
If this is the case and the simulation software belongs to some library,
modifications are generally avoided due to different reason such as: 
scarce knowledge of the implemented problem, old codes with 
obscure implementations difficult to modify, robust and validated codes
or standard libraries. That's why, generally,  
modifications are not desirable or not allowed. 
However, if extra 
functionalities such as:  debugging, profiling or plotting  
are not implemented, the elegant way of having those extra functionalities
without modifying the standard code is by means of wrappers or decorators. 
It is important to note that a wrapper or decorator can modify or complement 
the behavior of and existing code by means of watching input arguments and
processing output results. 
There two main uses of wrappers: 
\vspace{-0.5cm}
\begin{enumerate}
\setlength\itemsep{0cm}
\item Basic wrappers to communicate between to different libraries.
\item Advanced wrappers or decorators to extend 
functionality of existing software. This extra functionality could include:
 profiling, debugging, checking valid arguments, exception treatment and plotting. 
\end{enumerate} 
Besides, multiple decorators  can be used at the same time. A simulation software can
be decorated to validate input arguments, to measure its memory requirement and 
its time and, finally, to plot or show results in a graphical user interface. 

\newpage
\section{Basic wrapping}
The basic use of wrappers is to reuse old codes by wrapping them with a modern 
function interface. This basic use could covers the two following needs: 
\begin{enumerate}
\item To reduce the number of input arguments. 
If programs were written in the past where languages do not allow 
functionalities such as dynamic allocation, 
extra input arguments were defined to store working memory areas.
Wrappers allow to reduce those extra arguments by allocating dynamically 
those working areas before the old code is called.  

\item To adapt input arguments to match some required specifications.
If the existing software is going to be integrated in some specific library 
to be used by other modules, the function interface should 
adapted to new specifications. This can be done very easily 
by writing a wrapper 

\item To hide spaghetti or non structured codes by means of a nice wrapper. 
\end{enumerate}

\subsection*{Fortran} 
The Dormand–Prince method or DOPRI method
is an embedded Runge–Kutta ODE solver (Dormand \& Prince 1980). 
Dormand–Prince method is currently an option in Python's SciPy ODE integration library.
In the following example a call to an old numerical scheme \verb|DOPRI5| is shown.
The code was written by E. Hairer and G. Wanner 
and described in: "Differential Equations I. Nonstiff problems". 1993, Springer-Verlag. 
\vspace{0.5cm} 
   \renewcommand{\home}{./Fortran/sources/Advanced_programming/odes/schemes} 
   \lstfor
  \listings{\home/Temporal_Schemes.f90}{CALL DOPRI5}
  {IDID}{Temporal_Schemes.f90}
The subroutine \verb|DOPRI5| has 17 arguments. 
A wrapper is written to reduce the function interface to 5 arguments. 
\vspace{0.5cm}   
\listings{\home/Temporal_Schemes.f90}{subroutine WDOPRI5}
{WDOPRI5}{Temporal_Schemes.f90}


\newpage 
\section{Explaining decorators}
There is a specific syntactic sugar to write wrappers in Python. 
These wrappers are called decorators.
To explain how a decorator works, a recall to functional programming should be made. 
The important key to remember is that first class functions can  
have input arguments as functions and also can return functions. 
 

\subsection*{Python} 
In the following code, a function called \verb|simple_decorator| is shown.  
It has an input function argument \verb|f| and it returns the internal 
function \verb|wrapped_function|. 
\renewcommand{\home}{./Python/sources/Advanced_programming/decorators}
\lstpython 
\vspace{0.5cm}
\listingsp{\home/decorators.py}{simple_decorator}
  {return wrapped_function}{decorators.py}
In the following snippet, a function \verb|f(x)| is defined and 
a new function \verb|g| is created by means of \verb|simple_decorator| function. 
Since \verb|g=simple_decorator(f)|, \verb|g| is the internal new function \verb|wrapped_function| which prints its name and its input
argument, then it evaluates \verb|f(x)| and finally, prints its output. Hence, \verb|g| decorates \verb|f(x)|
with those input/output messages. Python has  syntactic sugar to make this creation easier. 
By positioning \verb|@simple_decorator| just above the definition of \verb|f(x)|, any invocation 
of \verb|f(x)| will be decorated having the same effect than \verb|g|
but simplifying the syntax code.    
\vspace{0.5cm}
\listingsp{\home/decorators.py}{@simple_decorator}
  {4}{decorators.py} 

\newpage 
\section{Debugging} 
When debugging code, it always important to check the type of input arguments,
its values and to know the returning values. An extra difficulty is that 
functions are defined with different 
numbers of input arguments. 
To overcome this difficulty, 
Python allows to define arguments by preceding its name with an asterisk (*), 
indicating argument tuple packing. 
Any corresponding arguments in the function call are packed into a tuple 
that the function can refer to by 
the given name.
Besides, Python has a similar operator, the double asterisk (**),  
indicating that the corresponding arguments are expected to be key=value pairs.

In the following code, a function \verb|debugging| prints the input values of any function 
\verb|f| either arguments are given by their value or when arguments are given by key-value pairs. 
After the function is called \verb|r=f(*args, **kwargs)|, the type of the result and its value 
is printed.  
\vspace{0.5cm}
\listingsp{\home/decorators.py}{Decorator to debug}
  {return debug_f}{decorators.py}

By positioning \verb|@debugging| just above any function definition, input arguments
and output value of any function will be printed.  



\newpage 
\section{Profiling} 
Program profiling is a dynamic analysis to measure memory consumption and CPU processing time 
involved when executing some programming code. 
Profiling information can be used to optimize the code by modifying those parts more 
demanding in erms of memory or CPU time. 
When functional programming paradigm is adopted, measuring isolated functions is more than enough 
to identify those critical demanding parts. 
To measure the CPU time and the memory consumption of some function, 
different libraries can be used to trace memory allocation and processing time. 

The following decorator \verb|profiling|  uses the \verb|tracemalloc| module  to trace memory blocks allocated by Python
and the \verb|time| module with the \verb|perf_counter| 
function to give  the float value of time in seconds. 
\vspace{0.5cm}
\listingsp{\home/decorators.py}{Decorator to measure}
  {return profile_f}{decorators.py}
By positioning the decorator \verb|@profiling| just above any function, used memory and 
elapsed time can be calculated.
The important idea of this function or decorator is that it is written only one time 
and it is used for any function to measure making codes much smaller and easier to maintain.


\newpage
\section{Exceptions}
Generally, functions are not prepared or protected for different combinations of input 
arguments. There are many situations when undesired combinations of input values gives 
strange or unforeseen behavior giving rise to math exceptions and other kind of 
malfunction or errors. When an error occurs, 
Python will stop and generate an error message.
These exceptions can be handled using the \verb|try| statement for a block of code. 
The \verb|except| statement for a block of code lets to handle the error.


Instead of writing  repetitive \verb|try/except| structure for any single function, the following 
decorator \verb|exceptions| can be used as a template 
to protect any function form possible malfunctions. 

\vspace{0.5cm}
\listingsp{\home/decorators.py}{exceptions}
  {return exception_treatment}{decorators.py}

In the following snippet, \verb|f| function is decorated with \verb|debugging|, 
\verb|profiling| and \verb|exceptions| at the same time. 
In the last evaluation of \verb|f("2",5)|, the \verb|exceptions| decorator handles the error
of trying to add string characters with integers numbers. 
\vspace{0.5cm}
\listingsp{\home/decorators.py}{@debugging}
  {exception f}{decorators.py}


\newpage 
\section{Decorated simulations} 
In this last example, four simulations of the Kepler orbit with different numerical schemes 
are decorated with: \verb|profiling| 
to measure used memory and CPU time, \verb|debugging| to monitor input arguments and 
\verb|plot_simulation| to plot the simulated orbit. 

\vspace{0.5cm}
\renewcommand{\home}{./Python/sources/Advanced_programming/ODES}

\listingsp{\home/main_ODES.py}{Decorator to plot}
  {return plot}{main_ODES.py}

\listingsp{\home/main_ODES.py}{Decorators to plot}
  {RK4}{main_ODES.py}