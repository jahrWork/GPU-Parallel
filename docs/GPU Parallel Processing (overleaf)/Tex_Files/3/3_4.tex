\subsection{Operators Used in Benchmark.}
    \subsubsection{CUDA}
    Compute Unified Device Architecture (CUDA) is a parallel computing platform and programming model developed by NVIDIA. It enables developers to leverage NVIDIA GPUs for general-purpose computing tasks beyond traditional graphics rendering. CUDA provides a set of extensions to standard programming languages such as C, C++, and Fortran, allowing developers to write programs that execute on the GPU. CUDA facilitates efficient parallel processing by exposing low-level GPU architecture features, including fine-grained control over memory and execution.

    \paragraph{Key Features of CUDA:}
    \begin{itemize}
        \item \textbf{Parallel Computing Model:} CUDA allows for the execution of multiple threads in parallel on the GPU. Threads are organized into blocks, and blocks are organized into grids, providing a flexible and scalable way to manage parallelism.
        \item \textbf{Memory Hierarchy:} CUDA provides access to different types of memory, including global, shared, and local memory, enabling optimization of memory access patterns and efficient data management.
        \item \textbf{Kernel Functions:} In CUDA, functions executed on the GPU are called kernels. Kernels are written in CUDA C/C++ and are launched from the host (CPU) to run on the device (GPU).
        \item \textbf{Thrust Library:} CUDA includes the Thrust library, which provides high-level abstractions for common parallel algorithms, such as sorting and reduction, simplifying development.
    \end{itemize}

    To use CUDA with Julia, you need to install the CUDA toolkit and the necessary Julia packages. 
    
    Once the CUDA toolkit and Julia packages are installed, you can start utilizing CUDA features in your Julia code. Here are some common tasks:

    \begin{itemize}
        \item \textbf{Checking GPU Availability:} Use the `CUDA` package to check if your GPU is available and properly configured:
        \begin{verbatim}
            using CUDA
            println(CUDA.has_cuda())
        \end{verbatim}
        \item \textbf{Allocating and Transferring Data:} Allocate memory on the GPU and transfer data between the host and device:
        \begin{verbatim}
            a = CUDA.rand(10)  # Allocate an array on the GPU
            b = CUDA.fill(2.0, 10)  # Fill an array with a constant value on the GPU
            c = a .+ b  # Perform element-wise addition on the GPU
        \end{verbatim}
    \end{itemize}

    \paragraph{CUDA Utilities:} In addition to core CUDA functionalities, several utilities and tools are available to help with development and performance optimization:
    \begin{itemize}
        \item \textbf{NVIDIA Nsight:} A suite of tools for debugging and profiling CUDA applications, providing insights into performance and helping identify bottlenecks.
        \item \textbf{CUDA Profiler:} Built into the CUDA toolkit, this tool allows for detailed performance analysis of GPU applications, including memory usage and execution time.
        \item \textbf{CUDA Samples:} The CUDA toolkit includes sample code and examples that demonstrate various features and best practices for CUDA programming.
    \end{itemize}

    Understanding and utilizing CUDA effectively can significantly enhance the performance of computational tasks by leveraging the parallel processing power of NVIDIA GPUs.