



%\setlength\parindent{0pt} % Removes all indentation from paragraphs
%\setlength{\parskip}{2ex} % Space between paragraphs
\setlength\parindent{18pt}
\setlength{\parskip}{\baselineskip}


%\topmargin 0.0cm
%\oddsidemargin 0.2cm
%\evensidemargin 0cm 
%\textwidth 15.5cm 
%\textheight 23cm
%\footskip 1.0cm


\definecolor{mygray}{gray}{0.8}
\definecolor{mygray2}{gray}{0.4}
\definecolor{mygreen}{rgb}{0.0, 0.42, 0.24}
\definecolor{myred}{rgb}{0.8, 0.0, 0.0}
\definecolor{myorange}{rgb}{0.91, 0.41, 0.17}
\definecolor{myblue}{rgb}{0.0, 0.44, 1.0}




\pagestyle{fancy} % Enables the custom headers/footers
\lhead{\small{Atentto manual}}
\chead[]{}
\fancyhead[R]{\leftmark}

%To make the head rule
\usepackage{etoolbox}
\makeatletter
\patchcmd{\headrule}{\hrule}{\color{mygray}\hrule}{}{}
\makeatother

%\setlength{\cftsecindent}{-0.8cm} % the section and the subsection in the table of contents
%\setlength{\cftsubsecindent}{-0.6cm} % the section and the subsection in the table of contents


% Replace the title of the table of contents 
\addto\captionsenglish{
	\renewcommand{\contentsname}%
	{Contents}% Here it goes the new title
}

% Here we give the color to the subsections in the table of contents

%\makeatletter
%\let\stdl@subsection\l@subsection
%\renewcommand*{\l@subsection}[2]{%
%	\stdl@subsection{\textcolor{blue}{#1}}{\textcolor{black}{#2}}}
%\makeatother

% CHANGING THE FORMAT OF SUBSECTIONS AND SECTIONS

%\titleformat\subsection{}{}{0em}{ \bf \todo[inline, color=green!40]}[\vspace{0ex}]



%\titleformat\section{\centering \LARGE}{}{0em}{ \bf }[]

%\renewcommand{\thesubsection}{} %Subsection numbering
%\renewcommand{\thesection}{} %Section numbering



%\newcommand{\bluepageref}[1]{\textcolor{blue}{\pageref{#1}}}
\newcommand{\qnameref}[1]{``\nameref{#1}''}



% Comment the following to have chapters numbered without interruption (numbering through parts)
\makeatletter\@addtoreset{chapter}{part}\makeatother%

% \tableofcontents, without the title contents
%\makeatletter
%\@starttoc{toc}
%\makeatother
%



\newcommand{\remark}[1]{\textbf{``#1"}}







%Allow us to make the rules larger

\renewcommand{\headrulewidth}{0.5pt} % No header rule
\renewcommand{\footrulewidth}{0.5pt} % Thin footer rule


%%PARA DOS CARAS
%\fancyhf{}				%allows additional selectors H (header) and F (footer)
\fancyhead[RO]{ \thepage}
\fancyhead[LE]{ \thepage}
%\fancyhead[LE,RO]{...}
\fancyhead[RE]{\sffamily\scshape\footnotesize\leftmark}
\fancyhead[LO]{\sffamily\scshape\footnotesize\rightmark}
\fancyfoot[CE,CO]{}
\fancyfoot[LE,RO]{}


%% Headers - all currently empty
%\lhead{}
%\chead{}
%\rhead{}
%% Footers
%\lfoot{}
%\cfoot{}
%%\rfoot{\footnotesize Page \thepage\ of \pageref{LastPage}} % "Page 1 of 2"
%\rfoot{\footnotesize \thepage\ }


%\usepackage{lettrine} % Package to accentuate the first letter of the text









%\newcommand{\tit}{Raspberry Pi Server}
%\newcommand{\rhdr}{Developer Manual}


%\DeclareCaptionFormat{myformat}{#1#2#3\vspace{2ex}}
%\captionsetup{format=myformat}

%\captionsetup[lstlisting]{position=bottom,format=myformat}
%\usepackage{lstautodedent}

%commands

%\newcommand{\relpathto}[1]{./codes/#1}

%\renewcommand{\medskip}{\vspace{2ex}}

%\newcommand{\capof}[2]{\begingroup\captionof{#1}{#2}\endgroup}


\newcommand{\onefigure}[3]{%%%%%%%%%%%%%%%%%
	\begin{figure}[h]
		\centering
		\captionsetup{width=0.5\textwidth}
		\begin{minipage}{0.5\textwidth}
			\includegraphics[width=\textwidth]{#1/#2}	
			%	\subcaption{}
		\end{minipage}
		\caption[#3]{#3}
\end{figure} }

\newcommand{\twofigures}[4]{%%%%%%%%%%%%%%%%%
	\begin{figure}[h]
		\centering
		\captionsetup{width=0.85\textwidth}
		\begin{minipage}{0.4\textwidth}
			\includegraphics[width=\textwidth]{#1/#2}	
			\subcaption{}
		\end{minipage}
		\hspace{0.05\textwidth}
		\begin{minipage}{0.4\textwidth}
			\includegraphics[width=\textwidth]{#1/#3}	
			\subcaption{}
		\end{minipage}
		\caption[#4]{#4}
\end{figure} }


\newcommand{\fourfigures}[6]{%%%%%%%%%%%%%%%%%
	\begin{figure}[h]
		\centering
		\captionsetup{width=0.85\textwidth}
		\begin{minipage}{0.4\textwidth}
			\includegraphics[width=\textwidth]{#1/#2}	
			\subcaption{}
		\end{minipage}
		\hspace{0.05\textwidth}
		\begin{minipage}{0.4\textwidth}
			\includegraphics[width=\textwidth]{#1/#3}	
			\subcaption{}
		\end{minipage}
		\begin{minipage}{0.4\textwidth}
			\includegraphics[width=\textwidth]{#1/#4}	
			\subcaption{}
		\end{minipage}
		\hspace{0.05\textwidth}
		\begin{minipage}{0.4\textwidth}
			\includegraphics[width=\textwidth]{#1/#5}
			\subcaption{}
		\end{minipage}  
		\caption[#6]{#6}
\end{figure} }




\newcommand{\wfigures}[2]
{
	\begin{figure}[h]
		\centering
		\captionsetup{width=0.85\textwidth}
		\begin{minipage}{0.4\textwidth}
			\includegraphics[width=\textwidth]{#1/#2}	
			\subcaption{}
		\end{minipage}
     \end{figure}
}


%%%%%%%%%%%%%%%%%%%%%%%%%%%%%%%%%%%%%%%%%%%%%%%%%%%%%%%%%%%%%%%%%%%%%%%%%%%
%%                           TCOLORBOX BOXES                             %%
%%%%%%%%%%%%%%%%%%%%%%%%%%%%%%%%%%%%%%%%%%%%%%%%%%%%%%%%%%%%%%%%%%%%%%%%%%%
\newtcolorbox{IN}
{colback=blue!5!white,colframe=Blue!75!black,sharp corners=northwest, fonttitle=\bfseries, title=Important Notice}

\newtcolorbox{INWarning}
{colback=red!5!white,colframe=Red!75!black,sharp corners=northwest, fonttitle=\bfseries, title=Important Notice}

\newtcolorbox{mybox}[1]
{colback=blue!5!white,colframe=Blue!75!black,sharp corners=northwest, fonttitle=\bfseries, title=#1}

\newtcolorbox{ExampleCode}[1]
{colback=myred!5!white,colframe=myred!75!black,sharp corners=northeast, fonttitle=\bfseries, title=Example Code: #1}

\newtcolorbox{Example}[1]
{colback=myred!5!white,colframe=myred!75!black,sharp corners=northeast, fonttitle=\bfseries, title=Example: #1}

